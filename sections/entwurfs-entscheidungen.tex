\setauthor{Abdulrahman Al Sabagh}
\begin{spacing}{1}

    \section{React Native}

    Grundsätzlich wurde React Native aus mehreren Gründen verwendet:
    \begin{itemize}
        \item React Native ist das Standardframework für cross-plattform-Lösungen bei der Firma Solvistas.
        \item Vorhandene Kenntnisse in den Sprachen Javascript bzw. Typescript
        \item Auftraggber forderte eine cross-plattform-Lösung für das Frontend, um IOS- Android- Betriebssysteme abzudecken
        \item Endergebnis von React Native ist eine native App
    \end{itemize}




    \subsection{Was ist eine cross-plattform-Lösung}

    \begin{quotation}
        ``
        Eine Cross-Platform App besteht aus einem einzigen Code,
        der jeweils in die native Systemsprache von Apple, Android \& Co. kompiliert wird.
        Dadurch erhält man eine App, die mit wenig Entwicklungsaufwand auf mehreren
        Betriebssystemen zur Verfügung steht, sich aber dennoch wie eine native App
        anfühlt.''
        \cite{cross-plattform}

    \end{quotation}


    \subsection{Warum wurde eine cross-plattform-Lösung angefordert}
    Der Client von Relaxoon besteht aus einer einfachen Applikation, die keine Business Logik
    und auch keine Performance kritische Funktionalitäten hat.
    Daher ist die Verwendung einer cross-plattform-Lösung sehr vorteilhaft,
    da alle Plattformen von einem Codebase gepflegt werden können.



    \subsection{Warum ist Relaxoon native und nicht hybrid}

    \textbf{hybride Applikationen}: \begin{quotation}
        ``
        Eine hybride App kombiniert die besten Elemente von [N]ativen und Web-Apps.
        Sie werden wie eine native App installiert, aber es ist eigentlich
        eine Web-App innerhalb des Endgeräts.
        Hybride Apps werden in den gängigsten Sprachen für die Web-App Entwicklung, wie z.B. HTML und CSS, programmiert. Dies bedeutet, dass sie auf verschiedenen Plattformen verwendet werden können.
        Obwohl sie in der Sprache der Webanwendung entwickelt wurden, haben sie die gleiche Fähigkeit wie native Apps, sich an verschiedene Geräte, wie ein Tablet, Smartphone usw. anzupassen.
        ''
        \cite{native-vs-hybrid}
    \end{quotation}
    \textbf{natvie Applikationen}: \begin{quotation}
        ``
        Eine native App ist eine Anwendung, die entwickelt wurde,
        um auf einer bestimmten Plattform oder einem bestimmten Endgerät zu arbeiten.
        Aus diesem Grund können native Apps mit den auf der jeweiligen Plattform
        installierten Betriebssystem[s]funktionen interagieren und diese nutzen.
        ''

        \cite{native-vs-hybrid}

    \end{quotation}

    Grundsätzlich sind native Applikationen  viel schneller und barrierefreier als hybride Applikationen
    \cite{native-vs-hybrid}



    \subsection{Alternativen für React Native}

    \subsubsection{Kotlin Multiplatform}
    \textbf{K}otlin \textbf{M}utli\textbf{P}lattform (KMP) bietet bessere Performance als React Native und hat auch eine modulare Integration.
    \newline
    \textbf{modulare Integration}: \begin{quotation}
        ``
        Probably the biggest benefit in favor of Kotlin Multiplatform is that it’s
        an SDK and not a framework. This means that teams with existing apps can
        simply add a module or migrate a small part to assess its viability without
        a huge commitment. This really helps Kotlin address the biggest deterrent
        when moving to a new codebase.
        ''
        \cite{kotlin-multiplatform-vs-react-native}
    \end{quotation}
    Allerdings ist das Problem davon, dass KMP noch immer im Beta ist und daher ist diese nicht stabil.
    Außerdem  ist die Community von dieser Alternative nicht so groß wie die Community von React Native.
    Gute Kenntnisse in Swift \textbf{U}ser \textbf{I}nterface (UI) und Android \textbf{S}oftware \textbf{D}evelopment \textbf{K}it (SDK) werden für die Verwendung von Kotlin Multiplatform auch verlangt.
    Grund dafür ist, dass KMP keine Bilbiotheken für UI-Elementen wie React Native oder Flutter bereitstellt, sondern die Syntax der OS-Technologie verwendet.
    \cite{kotlin-multiplatform-vs-react-native}
    \subsubsection{Flutter}

    Flutter an sich ist viel schneller als React Native.
    Außerdem unterstüzt Flutter die Betriebssysteme  Windows, Linux und Macos zusätzlich zu Web, Android und IOS.
    \cite{flutter-vs-react-native}
    \newline
    Das Problem bei Flutter ist, dass man gute Kenntnisse in der Programmiersprache ''Dart'' haben muss. \cite{flutter-vs-react-native}
    Für die Entwicklung von React Native sind gute Kenntnisse in den Programmiersprachen Javascript und Typescript,
    die wir bereits gelernt haben,
    gebraucht.
    Außerdem ist die Unterstützung von Macos, Linux und Windows für uns gar nicht relevant.


    \subsubsection{Xamarin}

    Xamarin ist eine C\# Framework, die für die Entwicklung von nativen cross-plattform-Lösungen zuständig ist.
    Die Performance von Xamarin ist viel besser als die Performance von React Native. \cite{xamarin-vs-react-native}

    Die Entwickler haben sich aus den Gründen, die bei Flutter bereites erwähnt wurden, für React Native entschieden.
    Außerdem gibt es beim Xamarin kein ''hot reloading'' und daher muss das
    Program bei jeder kleinen Änderung neugestartet werden.\cite{xamarin-vs-react-native}



    \subsection{Nutzwertanalyse für das Frontend-Framework}
    \begin{tabular}{|p{5cm} | p{2cm} | p{2cm} | p{2cm} | p{2cm} | }
        \hline
        Kriterien                      & React Native & Xamarin & Flutter & KMP  \\
        \hline
        Vorhandene Kenntinsse max.50\% & 50\%         & 0\%     & 0\%     & 10\% \\
        \hline
        Barrierfreiheit max. 30\%      & 15\%         & 5\%     & 30\%    & 20\% \\
        \hline
        Performance max. 20\%          & 5\%          & 20\%    & 15\%    & 10\% \\
        \hline
        Summe                          & 70\%         & 25\%    & 45\%    & 40\% \\
        \hline
    \end{tabular}





    \section{Strapi}

    \subsection{Was ist ein CMS}

    \begin{quotation}
        ``Ein Content-Management-System (CMS) ist eine Softwareanwendung,
        die es Benutzern ermöglicht, digitale Inhalte zu erstellen, zu bearbeiten, gemeinsam zu editieren,
        zu veröffentlichen und zu speichern. Content-Management-Systeme werden typischerweise für Enterprise
        Content Management (ECM) und Web Content Management (WCM) eingesetzt.''
        \cite{cms}
    \end{quotation}



    \subsection{Warum wurde ein CMS verwendet}
    Der Auftragsgeber wollte die Applikation so schnell wie möglich veröffentlichen,
    da die Anzahl der Apps, welche die gleichen Anwendungsfälle wie Relaxoon haben,
    nicht so groß ist.
    Daher ist die Verwendung eines fertigen CMSes viel schneller als
    die Implementierung eines Backends.

    \subsection{Was ist ein Headless-CMS}
    \begin{quotation}

        ``
        Ein Headless CMS ist sowohl eine Weiterentwicklung als
        auch eine Verknappung eines klassischen CMS. Dem System
        werden integrale Bestandteile genommen, um es für
        unterschiedlichste Ausgaben kompatibel zu machen.
        Das gelingt dadurch, dass Frontend und Backend in einem
        Headless CMS nicht mehr monolithisch miteinander verknüpft
        sind. Das fehlende Frontend ist auch der Grund,
        wieso derartige CMS-Systeme als
        „kopflos“ (englisch: „headless“) bezeichnet werden.
        '' \cite{headles-cms}
    \end{quotation}

    \subsection{Warum wurde ein Headless-CMS verwendet}
    \begin{itemize}
        \item Vorgabe von dem Auftragsgeber
        \item Andere Arten von CMSes generieren statische HTML Seiten, welche für Relaxoon gar nicht gebraucht werden,
              da Relaxoon eine Mobileapp ist
        \item die Firma Solvistas beschäftigt sich intensiv mit Data Science. Da man auf dem CMS mittels REST zugreifen kann,
              will Solvistas ein Datenzentrum aus diesem in der Zukunft erstellen, zusätzlich zur Verwendung von CMS für Data Science Zwecke.
    \end{itemize}
    \subsection{Warum Strapi und nicht Wordpress \textbf{AP}plication \textbf{I}nterface (API)}



    \subsubsection{Vorteile von Wordpress API}

    \begin{itemize}
        \item Es gibt sehr viele Einstellungen und Features
        \item Unterstüzt \textbf{S}earch \textbf{E}ngine \textbf{O}ptimization (SEO)
        \item guter Community Support
        \item Einfach zu verwenden
    \end{itemize}
    \cite{strapi-vs-wordpress}

    \subsubsection{Nachteile von Wordpress API}
    \begin{itemize}
        \item  Limitierte Flexibilität, da viele Plugins und Addons kostenpflichtig sind.
              die kostenlose Verwendung von einigen Addons und Plugins ist limitiert.
        \item Ist nicht geeignet für eine Software mit großer Skalierung
        \item Wordpress wird von vielen Menschen verwendet und deshalb ist es für die Hacker nicht unrelevant
        \item Die Benutzeroberfläche davon ist nicht gut designt und deshalb ist die Verwendung davon meistens sehr verwirrend und unangenehm zu bedienen
    \end{itemize}
    \cite{strapi-vs-wordpress}

    \subsubsection{Vorteile von Strapi}

    Im Gegensatz zu Wordpress, kann man mit Strapi beliebig viele Plugins und Addons
    verwenden. Außerdem gibt es eine eingebaute Authentifizierung bzw. Autorisierungsfunktion
    zusätzlich zur Unterstützung von 20 unterschiedlichen Sprachen und REST
    bzw. Graphql APIS.
    \cite{strapi-vs-wordpress}

    \subsubsection{Nachteile von Strapi}
    Ein Grundwissen in der Programmierung ist erforderlich, wenn man sich für Strapi
    entscheidet. Außerdem ist es nicht so weit verbreitet wie Wordpress und somit ist die Anzahl
    der 3rd-party-Libraries, die man in Strapi einbetten kann, nicht so groß.
    \cite{strapi-vs-wordpress}

    \subsubsection{Ergebnis}
    Alle erwähnten Pros und Contras zeigen, dass  die \textbf{U}ser E\textbf{X}perience (UX)
    und die Skalierbarkeit von Strapi viel besser als Wordpress ist.
    Außerdem ist Strapi viel schneller als Wordpress,
    da Strapi in "Node js" geschrieben ist und Wordpress in "PHP"
    \cite{strapi-vs-wordpress}

    \subsection{Nutzwertanalyse für das Headless-CMS}

    \begin{tabular}{ |p{3cm}|p{3cm}|p{3cm}| }
        \hline
        Kriterien                   & Strapi & Wordpress API \\
        \hline
        Security max.30\%           & 30\%   & 15\%          \\
        \hline
        Anpassbar max. 30\%         & 30\%   & 15\%          \\
        \hline
        Community Support max. 30\% & 10\%   & 25\%          \\
        \hline
        Performance max. 10\%       & 10\%   & 5\%           \\
        \hline
        Summe                       & 80\%   & 65\%          \\
        \hline
    \end{tabular}

\end{spacing}