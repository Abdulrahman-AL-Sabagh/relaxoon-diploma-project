\setauthor{Moritz Eder}

Es gibt einen klaren Unterschied zwischen dem Setzen von Zielen und dem tatsächlichen Erreichen von Zielen.
Das einfache Festlegen von Zielen allein genügt nicht. Um Erfolg in einem bestimmten Vorhaben zu haben,
ist es notwendig, klare Ziele zu definieren. Ein Ziel repräsentiert einen Zustand, ein Ergebnis oder einen
bestimmten Ort. Der Grad und die Art der Zielerreichung sind entscheidend für die Definition von Erfolg. Daher
ist das Festlegen eines Ziels lediglich der Ausgangspunkt. Dafür müssen Fortschritte oder Meilensteine
erreicht und der vorgegebene Zeitrahmen eingehalten werden – und dies alles unter Verwendung legaler und
legitimer Mittel und Methoden. Oftmals scheitert man nicht an mangelnder Disziplin oder Motivation, sondern
an falsch formulierten oder unpassenden Zielen.

Ziele charakterisieren sich durch einfache Aufgaben, die nach der Reihe erledigt werden – sie repräsentieren
feste Absichten. Hinter jedem Ziel steht immer ein konkretes Bestreben.
Ziele sind nicht nur das Ergebnis rationaler Überlegungen, sondern vor allem eine Angelegenheit des Herzens
und der Motivation.\cite{ziele}

\begin{figure}[H]
    \centering
    \includegraphics[height=0.45\textwidth]{./pics/undraw_Target_re_fi8j.png}
    \caption{}
\end{figure}

Ziele zu setzen ist wichtig, deswegen müssen sie von Anfang an \textbf{richtig} gesetzt werden. Beim Setzen der Ziele
kann man sich an folgenden Punkten orientieren: \cite{ziele}

\begin{itemize}
    \item Klare Konkretisierung ist entscheidend für Ziele! Je präziser die Beschreibung ist, desto einfacher
          ist es, darauf hinzuarbeiten.
    \item Ziele müssen realistisch sein! Zu ehrgeizige Ziele können Frustration und Selbstzweifel
          auslösen. Sowohl das angestrebte Ergebnis als auch die dafür vorgesehene Zeit sollten der Realität
          entsprechen.
    \item Wenn Ziele öffentlich gemacht werden, erhöht das auch die Erfolgswahrscheinlichkeit. Der Austausch
          über Ziele mit Familie, Freunden oder Kollegen steigert die Motivation und erzeugt Verbindlichkeit, da
          man es anderen beweisen möchte.
    \item Immer positiv bleiben! Positive Formulierungen der Ziele motivieren langfristig mehr.
    \item Ein eigener Antrieb ist notwendig, um die Zielabsicht vor Augen zu haben und um dranbleiben zu
          können.
    \item Ziele erfordern Flexibilität. Der Weg zum Ziel kann sich ändern, ebenso die Zeit für das Erreichen
          eines Meilensteins oder sogar das Ziel selbst. Ziele sind dynamisch und nie statisch oder unveränderlich.
    \item Zur Visualisierung der Ziele sollten sie stets in der Gegenwart formuliert werden, damit man sich
          gleich Bilder im Kopf vorstellen kann. Das steigert ebenso die Motivation.
\end{itemize}

\section{Projektanforderungen}

Beim Projektmanagement wird grundsätzlich zwischen zwei Hauptkategorien von Anforderungen unterschieden.

\subsection{Funktionale Anforderungen:}
\begin{itemize}
  \item Beschreiben, welche Funktionen das System bereitstellen soll und welche Aufgaben es ausführen muss.
  \item Definieren die spezifischen Funktionen, Dienste oder Aufgaben, die das System erfüllen muss, um die Benutzeranforderungen zu erfüllen.
  \item Konzentrieren sich auf die Was-Aspekte des Systems und beschreiben, welche Ergebnisse erwartet werden.
  \item Beispiel: "`Das System muss dem Benutzer ermöglichen, sich mit einem Benutzernamen und Passwort anzumelden."'
\end{itemize}

\subsection{Nicht funktionale Anforderungen:}
\begin{itemize}
  \item Beschreiben Eigenschaften des Systems, die nicht unbedingt Funktionen sind, sondern Qualitätsmerkmale, die das System erfüllen muss.
  \item Betreffen oft die Wie-Aspekte des Systems und definieren Qualitätsmerkmale wie Leistung, Sicherheit, Benutzerfreundlichkeit und Zuverlässigkeit.
  \item Berücksichtigen Aspekte wie Skalierbarkeit, Wartbarkeit, Zuverlässigkeit, Leistung und Benutzerfreundlichkeit.
  \item Beispiel: "`Das System muss eine Antwortzeit von weniger als 2 Sekunden für Benutzeranfragen sicherstellen."'
\end{itemize}

Der vorherige Text wurde von einem KI-Modell generiert.\cite{chatgpt}

\rule{\linewidth}{0.5pt}

Bei Relaxoon können alle Anforderungen auch in funktionale und nicht funktionale Anforderungen unterteilt werden.
Eine nicht funktionale Anforderung wäre zum Beispiel die einfache Bedienung der App. Der User oder die Userin
soll sich in der App problemlos zurechtfinden können, ohne dafür ein Manual zu benötigen. Dazu braucht man:
\begin{itemize}
      \item  Eine Erklärung am Anfang bzw. ein Tutorial, wenn man die App das erste Mal startet, um überhaupt 
      zu wissen, wie die App funktioniert und wie sie zu verwenden ist.
      \item Ein schlichtes und übersichtliches Registrierungsformular mit Begründung dazu, damit der User oder
      die Userin weiß, warum er sich registrieren muss.
      \item Eine Homepage auf der man sich leicht zurechtfindet und eine Übersicht über populäre Medien hat, die
      bereits hochgeladen worden sind.
      \item Eine Navigationsleiste mit der man einfach zu anderen Seiten oder Features der App weitergeleitet wird.
      \item Ein klarer Weg, um selbst personalisierte Einstellungen treffen zu können in der App. 
\end{itemize}

Auf der anderen Seite wäre ein Beispiel für eine funktionale Anforderung die Möglichkeit, sich wieder einloggen zu
können, wenn man sich bereits schonmal registriert hat, damit ein zu lange dauernder Login den User / die Userin
nicht verleitet, die App direkt wieder zu verlassen. Das kann man mit einer "`Login Daten merken"'-Option umsetzen
oder man verwendet eine "`Ich besitze bereits einen Account"'-Option bei User:innen, die sich die App auf einem
zusätzlichen Endgerät heruntergeladen haben. Eine weitere Projektanforderung für Relaxoon ist die
\textbf{Übertragbarkeit} - die App für mehrere Betriebsysteme nutzbar machen. Wenn eine App übertragbar ist,
schaffst das einige Vorteile:
\begin{itemize}
      \item Erhöhte Zielgruppenabdeckung
      \item Keine Vorgabe des Mobilgerätetyps
      \item Wettbewerbsvorteil gegenüber anderen
\end{itemize}

\section{Projektziele}

%%Zielhierarchie siehe Handyfoto

Bei den Projektzielen ist es wichtig sich nicht zu große Ziele vorzunehmen und nicht nur das eine Ziel im Auge zu haben.
Es wichtig ein großes Ziel aufzuteilen, sodass viele kurzfristige Ziele zu einem langfristigen Ziel führen können.
Langfristige Ziele bilden dabei auch die Leistungswirkung.

Relaxoon wurde in jeweils zweiwöchigen Sprints entwickelt. Es wurden am Anfang jedes Sprints einige 
kleine/kurzfristige Ziele definiert, die in den darauffolgenden Wochen erreicht werden sollten. Am Ende des 
Sprints wurde dann der damalige Stand des Projektes mit den Zielen verglichen, um zu wissen, ob alle Ziele 
erreicht wurden und welche Ziele für den nächsten Sprint vorgesehen waren.

Übersicht der wichtigsten kurzfristigen Ziele pro Sprint:
\begin{itemize}
      \item Sprint 0: Wahl des CMS zwischen Strapi und Wordpress
      \item Sprint 1: Strapi- sowie App-Setup fertigstellen, Beginn UI-Design der App
      \item Sprint 2: Zugriff von Frontend auf Backend möglich
      \item Sprint 3: Beginn Implementierung des Category-Screens
      \item Sprint 4: Datenmodell von Relaxoon mit Strapi abgleichen
      \item Sprint 5: Implementierung der Datenstrukur in Strapi
      \item Sprint 6: UI-Design in der App umsetzen, Entwicklung Suchleiste
      \item Sprint 7: Echte Daten des Backends am Frontend anzeigen
      \item Sprint 8: Implementierung des Registrierungsformulars und das Setzen eines Favoriten 
      \item Sprint 9: Anmeldung in der App ermöglichen, Umsetzung Light und Dark Mode
      \item Sprint 10: Auto-Login, Tags für Medien hinzufügen
\end{itemize}

Faktoren wie Zeit, Kosten, Nutzen und Qualität sollten ebenfalls bei jedem Ziel berücksichtigt werden. Man
sollte darauf achten, dass sich diese Faktoren gut miteinander vereinen lassen:
\begin{itemize}
      \item Zeit: Kann das Ziel in der geplanten Zeit erreicht werden?
      \item Kosten: Welche Kosten werden der Umsetzung anfallen?
      \item Nutzen: Was wird der Nutzen dieses Zieles sein? Für welchen Zweck wird es gebraucht?
      \item Qualität: In welcher Qualität, soll das Ziel umgesetzt werden?
\end{itemize}

\subsection{Leistungswirkungen}

Leistungswirkungen eines Projektes beschreiben nicht das Ziel an sich, sondern was aus den Zielen folgt.
Die Entspannungsapp Relaxoon soll bei den Usern und Userinnen der App für geringen Stress und eine hohe
Stressreduktion sorgen, was die Personen produktiver und glücklicher macht. Eine langfristige Erhaltung oder
Verbesserung der Gesundheit wäre eine ideale Leistungswirkung. 

Um die formulierten Ziele in gewünschter Qualität und fristgerecht zu erreichen, mussten
neben Schul- bzw. Lernanforderungen die Ausdauer und Priorität auf das Projekt
gesetzt werden.
