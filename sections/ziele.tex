\setauthor{Moritz Eder}

Es gibt einen klaren Unterschiedzwischen dem Setzen von Zielen und dem tatsächlichen Erreichen von Zielen.
Das einfache Festlegen von Zielen allein genügt nicht. Um Erfolg in einem bestimmten Vorhaben zu haben,
ist es notwendig, klare Ziele zu definieren. Ein Ziel repräsentiert einen Zustand, ein Ergebnis oder einen
bestimmten Ort. Der Grad und die Art der Zielerreichung sind entscheidend für die Definition von Erfolg. Daher
ist das Festlegen eines Ziels lediglich der Ausgangspunkt. Dafür müssen Fortschritte oder Meilensteine
erreicht und der vorgegebene Zeitrahmen eingehalten werden – und dies alles unter Verwendung legaler und
legitimer Mittel und Methoden. Oftmals scheitert man nicht an mangelnder Disziplin oder Motivation, sondern
an falsch formulierten oder unpassenden Zielen.

Zielen charakterisieren sich durch einfache Aufgaben, die nach der Reihe erledigt werden – sie repräsentieren
feste Absichten. Hinter jedem Ziel steht immer ein konkretes Bestreben.
Ziele sind nicht nur das Ergebnis rationaler Überlegungen, sondern vor allem eine Angelegenheit des Herzens
und der Motivation.\cite{ziele}

\begin{figure}[H]
    \centering
    \includegraphics[height=0.45\textwidth]{./pics/undraw_Target_re_fi8j.png}
    \caption{}
\end{figure}

Ziele zu setzen ist wichtig, deswegen müssen sie von Anfang an \textbf{richtig} gesetzt werden. Beim Setzen der Ziele
kann man sich an folgenden Punkten orientieren: \cite{ziele}

\begin{itemize}
    \item Klare Konkretisierung ist entscheidend für Ziele! Je präziser die Beschreibung ist, desto einfacher
          ist es, darauf hinzuarbeiten.
    \item Ziele müssen realistisch sein! Zu ehrgeizige Ziele können Frustration und Selbstzweifel
          auslösen. Sowohl das angestrebte Ergebnis als auch die dafür vorgesehene Zeit sollten der Realität
          entsprechen.
    \item Wenn Ziele öffentlich gemacht werden, erhöht das auch die Erfolgswahrscheinlichkeit. Der Austausch
          über Ziele mit Familie, Freunden oder Kollegen steigert die Motivation und erzeugt Verbindlichkeit, da
          man es anderen beweisen möchte.
    \item Immer positiv bleiben! Positive Formulierungen der Ziele motivieren langfristig mehr.
    \item Ein eigener Antrieb ist notwendig, um die Zielabsicht vor Augen zu haben und um dranbleiben zu
          können.
    \item Ziele erfordern Flexibilität. Der Weg zum Ziel kann sich ändern, ebenso die Zeit für das Erreichen
          eines Meilensteins oder sogar das Ziel selbst. Ziele sind dynamisch und nie statisch oder unveränderlich.
    \item Zur Visualisierung der Ziele sollten sie stets in der Gegenwart formuliert werden, damit man sich
          gleich Bilder im Kopf vorstellen kann. Das steigert ebenso die Motivation.
\end{itemize}

\section{Projektziele}
Das Projekt soll zeigen, dass unter Verwendung des Open-Source headless CMS 'Strapi' und des mobilen 
React-Native Frontends einfach eine App zur Wiedergabe von Mediendateien für verschiedene mobile Betriebssysteme 
erstellt werden kann.

Ziel ist auch die Feinabstimmung der Qualität der Inhalte und wie diese optimal bereitgestellt werden können.
Wichtig ist dabei auch das für den Zweck passende Look and Feel.

