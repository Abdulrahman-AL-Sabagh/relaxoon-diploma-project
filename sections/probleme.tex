\setauthor{Abdulrahman Al Sabagh}
\begin{spacing}{1}


    \section{Probleme mit localhost und https}\label{sec:probleme-mit-localhost-und-https}

    Im Projekt wurde  die \textbf{U}niform \textbf{R}esource
    \textbf{L}ocater (URL) für den (API)-Server in einem ".env" File gespeichert.
    Für die lokale Entwicklung wurde Strapi auf dem lokalen Host immer verwendet.
    Bei IOS und Android ist das Holen der Daten von einer nicht sicheren Quellen
    \textbf{A}lso \textbf{K}nown \textbf{A}s (aka) nur http Protokolle gar nicht möglich. Auf Android gibt aber einige
    Ausnahmen für das Holen der Daten von dem lokalen Host
    Bei Android kann man bei den generierten Build und Project Files einige Konfigurationen ändern.
    Das ist zwar keine gute Lösung, da diese Files bei jedem Build geändert werden können.
    Außerdem können diese generierten Files nicht in das \textbf{V}ersion \textbf{C}ontrol \textbf{S}ystem (VCS)
    eingecheckt werden.
    Bei Android muss man 10.0.2.2 statt localhost schreiben.\cite{androidFetch}
    Bei IOS gibt es keine einzige Möglichkeit,
    um Daten aus http Quellen oder aus dem localhost zu holen.
    Als Lösung wurde in der lokalen Entwicklung eine \textbf{C}ommand \textbf{L}ine \textbf{I}nterface (CLI) namens "ngrok" verwendet,
    welche den lokalen Port des Servers ins Internet mittels eines Tunnels weiterleitet
    und eine https Adresse für den weitergeleiteten Server generiert.

    Das Problem tauchte wiederholt in der Test-Phase auf und war für die Entwickler nicht lösbar,
    da der Staging Server für das Backend von Relaxoon auf den Servern der Solvistas deployt ist.
    Die Entwickler verfügen über keine Zugriffsrechte auf diesem Server.
    Das Problem wurde mithilfe von den Netzwerkadministratoren von Solvistas gelöst.
    Es wurde ein "Let's encrypt" Zertifikat für den Staging Server eingebaut.





    \section{Inkompatible Libraries beim Build-Prozess}\label{sec:inkompatible-libraries-beim-build-prozess}
    In der Entwicklungsphase wurden einige Libraries mittels \textbf{N}ode  \textbf{P}ackage \textbf{M}anager (npm) installiert.
    Einige Libraries waren mit der Node Version oder mit anderen Libraries nicht kompatibel.
    Eine Lösung wäre, das Label "--legacy-peer-deps" oder "--force" beim Befehl "npm install" anzuhängen.
    Während der Entwicklung wurde das Label "--legacy-peer-deps" immer beim “npm install" integriert
    und danach hat alles problemlos funktioniert.
    Später beim Deployment von Android ist das Problem
    bei der Generierung von Android apk bzw.
    \textbf{A}ndroid \textbf{A}pp \textbf{B}undle (aab) aufgetaucht.
    Die Apk bzw.
    Aab wurde erfolgreich generiert, aber die App war nicht funktionsfähig.
    Es wurde für das Finden des Problems auf Android ein Tool namens “logcat” verwendet,
    um die Fehlermeldung zu finden.
    Dieses Problem an sich ist in der Community sehr stark verbreitet und es gibt dafür mehrere Gründe und mehrere Lösungen.
    Unter diesem Githublink \cite{libjsexecutor} werden unterschiedliche Gründe und mehrere
    Lösungsmöglichkeiten für das gleiche Problem beschrieben.
    Es wurde jede einzelne Lösungsmöglichkeit versucht und keine von diesen vorgeschlagenen Lösungen
    funktionierte.
    Aus Erfahrung hat man gewusst, dass der Codebase einige Libraries hat,
    die nicht mehr gebraucht bzw.
    verwendet werden. Beim Löschen dieser Libraries und die Wiederinstallation
    von den Modulen mittels "npm install" tauchte diese "legacy-peer-deps" Meldung wieder auf.
    Man dachte, dass das Problem daran liegt und es wurde  ein Tool namens "depcheck" installiert,
    welches den Codebase scannt und den Entwickler bekannt gibt, ob eine Library im Codebase verwendet
    wird oder nicht. \cite{depcheck}
    Nach dem Löschen aller nicht benötigten Libraries und die Generierung der
    apk bzw.
    aab hat die Applikation problemlos funktioniert.



    \subsection{npm install --legacy-peer-deps}\label{subsec:npm-install---legacy-peer-deps}
    \begin{quotation}
        ``
        The '--legacy-peer-deps' flag is used when you encounter compatibility issues with peer dependencies
        while installing packages.
        Peer dependencies are required by a package but aren't automatically
        installed alongside it.
        In some cases, when a package has not been updated to support the latest
        version of its peer dependency, the installation may fail due to conflicting versions.
        Adding the '--legacy-peer-deps' flag allows npm to use an older, compatible version of
        the peer dependency, ensuring a successful installation.''
        \cite{installFlags}
    \end{quotation}

    \subsection{npm install --force}\label{subsec:npm-install---force}
    \begin{quotation}
        ``
        The '--force'  flag is a more drastic option and should be used with caution.
        It instructs npm to forcefully install packages, even if it encounters errors or conflicts.
        This can be useful in situations where you want to override any version or compatibility checks
        and forcibly install packages.
        However, it is important to note that using '--force'
        may lead to unexpected issues, such as breaking dependencies or introducing incompatibilities,
        so it should be used sparingly and with a good understanding of its consequences.''
        ~\cite{installFlags}
    \end{quotation}



    \section{Probleme mit Thumbnails}\label{sec:probleme-mit-thumbnails}


    Unsere App soll mehrere Videos für Antistress-Meditationen anzeigen.
    Jedes Video muss unbedingt ein Thumbnail haben.
    Der Content-Manager könnte aber beim Erstellen des Videos vergessen,
    ein Thumbnail für das Video zu erstellen.
    Auf unserer Adminoberflächen können keine Thumbnails zu dem Video hinzugefügt werden,
    da diese Aktion sehr schlechte User Experience verursachen könnte.
    Man kann zwar eine Funktion implementieren,
    die nach dem Hochladen eines Videos ein Thumbnail mittels \textbf{F}ast \textbf{F}orward \textbf{M}oving
    \textbf{P}icture \textbf{E}xperts \textbf{G}roup (ffmpeg) aus dem ersten Bild des Videos erstellt.
    Diese Möglichkeit war aber sehr aufwendig und nicht empfehlenswert, da die damalige Dokumentation von
    Strapi nicht sehr genau war.
    Außerdem könnte diese Alternative bei den Updates von Dependencies nicht mehr funktionsfähig sein.
    Zum Glück haben wir eine Expo-Library gefunden, die im Frontend Thumbnails für unsere Videos erstellt.
    Die Verwendung davon war aber nicht sehr vorteilhaft, da der Generierungsprozess sehr langsam war.
    Als Lösung haben wir uns entschieden, ein nicht abspielbares Video auf den Screens, wo mehrere Videos vorgeschlagen werden,
    anzuzeigen.
    Das Video ist erst abspielbar, wenn man darauf klickt.
    Wir haben diese Lösung mit den Konzepten "Lazy Loading" und "Suspenses" zusammen kombiniert,
    damit Loading-Spinners statt leere Flächen für den User angezeigt werden,
    wenn das Laden des Videos etwas länger dauert.



    \subsection{Suspense}\label{subsec:suspense}

    \begin{quotation}

        ``<Suspense> lets you display a fallback until its children have finished
        loading.''~\cite{suspense}
    \end{quotation}

    \subsection{Lazy Loading}\label{subsec:lazy-loading}

    \begin{quotation}
        ``lazy lets you defer loading component’s code until it is rendered for
        the first time.''~\cite{lazyLoading}
    \end{quotation}


    \section{Dauer von Videos}\label{sec:dauer-von-videos}

    Beim Hochladen eines Videos in Strapi,
    wird die Dauer des Videos nur auf der Oberfläche angezeigt.
    Diese Information ist aber in den REST-Schnittstellen der Meta-Daten von den Files nicht inkludiert.
    Bei diesem Problem ist die Verwendung von ffmpeg auch eine Lösungsmöglichkeit.
    Es wurde aber nicht mit ffmpeg gearbeitet (siehe das obige Problem).
    Für die Berechnung der Dauers wurde im Frontend das "onLoad" Event,
    welches in den Properties (props) des Videoelements ist, verwendet,
    um die Dauer zu lesen und diese in das Format "mm:ss" umwandeln zu können.





    \section{Probleme mit useEffect und React-navigation Library}\label{sec:probleme-mit-useeffect-und-react-navigation-library}
    Für das Holen der Daten aus dem Server würde die Fetch API verwendet.
    Diese wurde auch in einem "useEffect Hook" eingegeben, damit die Daten bei jeder Aktualisierung eines spezifischen
    Zustandes aus dem Server geholt werden können.
    Beim Anklicken der unterschiedlichen Navigationsbuttons wurde bemerkt,
    dass die Daten sich gar nicht ändern.
    Am Anfang wurde vermutet, dass das Problem ein Caching Problem sein könnte.
    Nach einer Recherche kam man darauf, dass die Library "React-navigation" den "useEffect Hook" im Hintergrund deaktiviert.
    Für das Holen der Daten ist die Übergabe ein sogenanntes "useCallback Hook" als Parameter in einem custom hook namens "useFocusEffect"
    von React-navigation notwendig.\cite{issuesWithUseEffect}


    \subsection{Hooks}\label{subsec:hooks}
    \begin{quotation}
        ``Hooks let you use different React features from your components.
        You can either use the built-in Hooks or combine them to build your own.
        This page lists all built-in Hooks in React.''
        \cite{hooks}
    \end{quotation}

    \subsection{useEffect}\label{subsec:useeffect}
    \begin{quotation}
        ``useEffect is a React Hook that lets you synchronize a component
        with an external system.''~\cite{useEffect}

    \end{quotation}
    \begin{quotation}
        ``Some components need to synchronize with external systems.
        For example, you might want to control a non-React component based on the React state, set up a server connection,
        or send an analytics log when a component appears on the screen.
        Effects let you run some code after rendering so that you can synchronize your component with some system outside of React.[...]
        Effects let you specify side effects that are caused by rendering itself,
        rather than by a particular event. ''
        \cite{Synchronizing-with-effects}


    \end{quotation}


    \subsection{useCallback}\label{subsec:usecallback}
    \begin{quotation}
        ``useCallback is a React Hook that lets you cache a function definition between re-renders.''
        \cite{useCallback}
    \end{quotation}


    \subsection{useFocusEffect}\label{subsec:usefocuseffect}
    \begin{quotation}
        ``The useFocusEffect is analogous to React's useEffect hook. The only difference is that it only runs if the screen
        is currently focused.
        The effect will run whenever the dependencies passed to React.useCallback change,
        i.e. it'll run on initial render (if the screen is focused) as well as on subsequent renders if the dependencies
        have changed. If you don't wrap your effect in React.useCallback, the effect will run every render if the screen
        is focused.'' \cite{useFocusEffect}

    \end{quotation}


\end{spacing}


