\setauthor{Moritz Eder}

Beim Projektmanagement wird grundsätzlich zwischen zwei Hauptkategorien von Anforderungen unterschieden.

\section{Funktionale Anforderungen:}
\begin{itemize}
  \item beschreiben, welche Funktionen das System bereitstellen soll und welche Aufgaben es ausführen muss.
  \item definieren die spezifischen Funktionen, Dienste oder Aufgaben, die das System erfüllen muss, um die Benutzeranforderungen zu erfüllen.
  \item konzentrieren sich auf die Was-Aspekte des Systems und beschreiben, welche Ergebnisse erwartet werden.
  \item Beispiel: "`Das System muss dem Benutzer ermöglichen, sich mit einem Benutzernamen und Passwort anzumelden."'
\end{itemize}

\section{Nicht funktionale Anforderungen:}
\begin{itemize}
  \item beschreiben Eigenschaften des Systems, die nicht unbedingt Funktionen sind, sondern Qualitätsmerkmale, die das System erfüllen muss.
  \item betreffen oft die Wie-Aspekte des Systems und definieren Qualitätsmerkmale wie Leistung, Sicherheit, Benutzerfreundlichkeit und Zuverlässigkeit.
  \item berücksichtigen Aspekte wie Skalierbarkeit, Wartbarkeit, Zuverlässigkeit, Leistung und Benutzerfreundlichkeit.
  \item Beispiel: "`Das System muss eine Antwortzeit von weniger als 2 Sekunden für Benutzeranfragen sicherstellen."'
\end{itemize}

Die oberen zwei Abschnitte 6.1 und 6.2 wurden von einem KI-Modell generiert.\cite{chatgpt}

\rule{\linewidth}{0.5pt}

Bei Relaxoon können alle Anforderungen auch in funktionale und nicht funktionale Anforderungen unterteilt werden.
Eine nicht funktionale Anforderung wäre zum Beispiel die einfache Bedienung der App. Der User oder die Userin
soll sich in der App problemlos zurechtfinden können, ohne dafür ein Manual zu benötigen. Dazu braucht man:
\begin{itemize}
      \item eine Erklärung am Anfang bzw. ein Tutorial, wenn man die App das erste Mal startet, um überhaupt 
      zu wissen, wie die App funktioniert und wie sie zu verwenden ist.
      \item ein schlichtes und übersichtliches Registrierungsformular mit Begründung dazu, damit der User oder
      die Userin weiß, warum er sich registrieren muss.
      \item eine Homepage, auf der man sich leicht zurechtfindet und eine Übersicht über populäre Medien hat, die
      bereits hochgeladen worden sind.
      \item eine Navigationsleiste, mit der man einfach zu anderen Seiten oder Features der App weitergeleitet wird.
      \item einen klaren Weg, um selbst personalisierte Einstellungen in der App treffen zu können. 
\end{itemize}

Auf der anderen Seite wäre ein Beispiel für eine funktionale Anforderung die Möglichkeit, sich wieder einloggen zu
können, wenn man sich bereits schonmal registriert hat, damit ein zu lange dauernder Login die User:innen
nicht verleitet, die App direkt wieder zu verlassen. Das kann man mit einer "`Login Daten merken"'-Option umsetzen
oder man verwendet eine "`Ich besitze bereits einen Account"'-Option bei User:innen, die sich die App auf einem
zusätzlichen Endgerät heruntergeladen haben. Eine weitere Projektanforderung für Relaxoon ist die
\textbf{Übertragbarkeit} d.h. die App für mehrere Betriebssysteme nutzbar machen. Wenn eine App übertragbar ist,
schaffst das einige Vorteile:
\begin{itemize}
      \item erhöhte Zielgruppenabdeckung
      \item keine Vorgabe des Mobilgerätetyps
      \item Wettbewerbsvorteil gegenüber anderen
\end{itemize}