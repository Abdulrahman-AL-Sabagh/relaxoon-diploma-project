\setauthor{Abdulrahman Al Sabagh}
\section{Strapi}
Strapi ist ein Headless \textbf{C}ontent \textbf{M}anagment \textbf{S}ystem (CMS),
welches eine vorgefertigte Benutzeroberfläche für Content-Creator
und auch für die Entwickler bereitstellt.

Mit der Verwendung davon sind Inhalte und dazu gebrauchte technische Funktionalitäten
sehr einfach erstellbar.
\cite{strapi-vs-wordpress}

Beispiele dafür sind:

\begin{itemize}
    \item \textbf{RE}presentational \textbf{S}tate \textbf{T}ransfer (REST)- bzw. \textbf{G}raph \textbf{Q}uery \textbf{L}anguage (Graphql)-Schnittstellen
    \item Logik für die Authentifizierung
    \item \textbf{C}reate, \textbf{R}ead, \textbf{U}pdate und \textbf{D}elete (CRUD)
          Funktionalitäten jeder Business Entität
\end{itemize}


\section{Firebase App Distribution}

``
Firebase App Distribution macht die Verteilung
Ihrer Apps an vertrauenswürdige Tester problemlos. Indem Sie Ihre Apps schnell
auf die Geräte der Tester übertragen, können Sie frühzeitig und häufig
Feedback einholen.
Und wenn Sie Crashlytics in Ihren Apps verwenden, erhalten Sie automatisch Stabilitätsmetriken für alle Ihre Builds, sodass Sie wissen, wann Sie zur Auslieferung bereit sind.
''\cite{fire-base-app-distribution}

Der Vorteil von Firebase App Distribution ist,
dass man die Applikation auf dem eigenen Mobilegerät ausprobieren kann,
ohne die App auf dem Play Store bzw. App Store hochladen zu müssen.