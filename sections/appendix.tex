\section{Prompt-Verlauf}

\textbf{Eingabe:} 

Erkläre mir den Unterschied zwischen funktionalen und nicht funktionalen Anforderungen.

\textbf{KI-Modell:} 

Funktionale und nicht-funktionale Anforderungen sind zwei Hauptkategorien von Anforderungen in der Softwareentwicklung und anderen Bereichen des Projektmanagements. Hier sind die grundlegenden Unterschiede zwischen ihnen:

Funktionale Anforderungen:
Beschreiben, welche Funktionen das System bereitstellen soll und welche Aufgaben es ausführen muss.
Definieren die spezifischen Funktionen, Dienste oder Aufgaben, die das System erfüllen muss, um die Benutzeranforderungen zu erfüllen.
Konzentrieren sich auf die "Was"-Aspekte des Systems und beschreiben, welche Ergebnisse erwartet werden.
Beispiel: "Das System muss dem Benutzer ermöglichen, sich mit einem Benutzernamen und Passwort anzumelden."

Nicht-funktionale Anforderungen:
Beschreiben Eigenschaften des Systems, die nicht unbedingt Funktionen sind, sondern Qualitätsmerkmale, die das System erfüllen muss.
Betreffen oft die "Wie"-Aspekte des Systems und definieren Qualitätsmerkmale wie Leistung, Sicherheit, Benutzerfreundlichkeit und Zuverlässigkeit.
Berücksichtigen Aspekte wie Skalierbarkeit, Wartbarkeit, Zuverlässigkeit, Leistung und Benutzerfreundlichkeit.
Beispiel: "Das System muss eine Antwortzeit von weniger als 2 Sekunden für Benutzeranfragen sicherstellen."

In einem Softwareprojekt sind funktionale Anforderungen eher auf die eigentlichen Aufgaben und Funktionen der Software ausgerichtet, während nicht-funktionale Anforderungen die Qualität und Leistung des Systems definieren. Beide Arten von Anforderungen sind wichtig, um ein vollständiges und erfolgreiches Produkt zu entwickeln, da sie gemeinsam die grundlegenden Spezifikationen und Erwartungen für das System festlegen.

\rule{\linewidth}{0.5pt}

