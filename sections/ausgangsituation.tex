\setauthor{Moritz Eder}
\begin{spacing}{1}
    
    Viele Menschen empfinden es als angenehm oder entspannen Musik zu hören. Bei dem heutzutage immer größer 
    werdenden Stress verliert man manchmal den Überblick über seine Aufgaben. Deshalb greifen viele Menschen
    auf die Musik zurück, wodurch sie sich beim Arbeiten entweder besser konzentrieren oder kurz 
    eine Pause einlegen können.

    Mit Stress und Leistungsdruck umgehen zu können, ist für Schülerinnen und Schüler auch extrem wichtig, um 
    nicht schon in ein frühes Burn-Out zu fallen. 
    
    Mögliche Folgen von hohem Stress oder hohem Leistungsdruck:
    %%https://www.internisten-im-netz.de/fachgebiete/psyche-koerper/stress.html#:~:text=Zeichen%20von%20Nervosit%C3%A4t%20(Z%C3%A4hneknirschen%20in,Stoffwechselst%C3%B6rungen%2C%20Allergien%20und%20Entz%C3%BCndungskrankheiten%20f%C3%BChren.)

    \begin{itemize}
        \item erhöhte Nervosität
        \item Stottern
        \item Vergesslichkeit
        \item Depression
        \item psychische Störungen
    \end{itemize}

    Anhaltender Stress kann sogar Herz/Kreislauf- und Nierenerkrankungen, Stoffwechselstörungen, Allergien oder
    Entzündungskrankheiten hervorrufen.

    Ein Mensch braucht also Ruhe und Entspannung um auch auf lange Zeit gut funktionieren und produktiv 
    lernen oder arbeiten zu können. 

    Zur Stressbewältigung wurde von der MACOLUTION GmbH eine Idee zur Lösung dieses Problem entwickelt. 

\end{spacing}