"`
Marktanalysen sind die Grundlage für wichtige strategische Geschäftsentscheidungen. 
Nicht nur Standortentscheidungen und die Wahl von Marketingmaßnahmen hängen davon ab, sie können – je nach 
Schwerpunkt – auch ausschlaggebende Informationen zur Preispolitik, Expansionsplänen und Produktentwicklung 
beisteuern. So helfen sie etwa dabei, dass ein Markteintritt oder die Neueinführung eines Produktes gelingt
oder ganz neue Märkte erschlossen werden können. 
"'
\cite{marktanalyse}

\section{Vorgehensweise}

In einem ersten Schritt wird der Markt und die Zielgruppe definiert. 
Ist die App global verfügbar oder nur in bestimmten Regionen? Wer sind die potenziellen Benutzer:innen?
Wie viel Interesse besteht an der App von den Kund:innen?

In einem nächsten Schritt muss der Wettbewerb untersucht werden.
Welche Entspannungsapps sind auf dem Markt, welche davon sind am erfolgreichsten und warum?
Dazu kommt noch eine Bewertung der aktuellen Trends im Bereich Gesundheit und Wohlbefinden. 
Gibt es aktuelle / neue Technologien oder Forschungsergebnisse, die in die App integriert werden könnten?

Danach werden verschiedene Geschäftsmodelle in Erwägung gezogen. 
Soll die App kostenlos sein und durch Anzeigen oder In-App-Käufe finanziert werden? 
Oder wird es eine kostenpflichtige Premium-Version geben, mit der man mehr Funktionen innerhalb der App freischaltet?

Schließlich muss eine Marketingstrategie entwickelt werden, um die Entspannungsapp bekannt zu machen und
in Umlauf zu bringen. 
Welche Kanäle sind am besten dazu geeignet, die gewünschten Zielgruppen zu erreichen?


%% Grafik zur Visualisierung mit Figma?

Durch diese Analyse erhält man wichtige Erkenntnisse über die aktuelle Situation sowie potenzielle Enwticklungen
auf der Markt für Entspannungsapps.


\section{Analyse}

Wie schon bei der Planung des Projektes formuliert war, soll die App Relaxoon im Play Store und im App Store 
international verfügbar sein, welche Stress bei bedürftigen Personen reduziert. Der Markt bei Entspannungsapps
ist jedoch groß, was zu hoher Konkurrenz führt. 

Einerseits gibt es viele Content-Creator, die ihre selbst erstellen Entspannungsmedien und -übungen auf den bekanntesten und
auch größten Plattformen wie zum Beispiel Youtube oder Spotify direkt hochladen, ohne die Verwendung einer App.
Das könnte bei einigen Menschen dazu führen, sich gar keine App installieren zu wollen, weil es unnötig erscheint.
Andererseits bieten Apps, die extra auf Entspannung konzipiert sind massive Vorteile. 

Einige markante Features von Entspannungsapps sind:

\begin{itemize}
    \item Punkte 1...
\end{itemize}

Bei Entspannungsapps gibt es auch verschiedene Arten. Zum einen welche, die sich ausschließlich auf Meditation
spezialisiert haben, wohingegen andere nur audiovisuelle Inhalte zur Verfügung stellen. Relaxoon soll deshalb
die Möglichkeit bieten, alle Bedürfnisse der User:innen abzudecken, wobei man sich persönlich als User:in 
auch dafür entscheiden kann, nach bestimmten Vorgaben zu filtern.

